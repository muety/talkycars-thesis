This chapter first outlines limitations of current approaches in the field of cooperative perception and motivates this work's contributions to overcome them. Second, goals and conceptual requirements are presented as a guideline for later system design. Finally, it is made clear which, aspects are in or out of scope of this thesis.

\section{Current Limitations}
\label{sec:problem_analysis:current_limitations}

Existing work in the field of cooperative perception, as presented in \autoref{ch:related_work}, faces several limitations.
\par
\bigskip

One of them is a \textbf{lack of comprehensiveness}. While the shown studies each focus on certain aspects of a CP system, to the best of our knowledge, no solution was presented that is both holistic and detailedly concerned with all individual parts of the system. Also, not all previous project provide an actual implementation of their proposal or do not conduct realistic simulations.
\par
\bigskip

Moreover, \textbf{no standardized, uniform, yet expressive environment model} and state representation for traffic scenes exists, yet. While the efforts taken by \cite{EuropeanTelecommunicationsStandardsInstituteETSI2019} to specify a standard for CPMs are promising, at this point, there is nothing like that available to be used in CP systems. Current models are either incomplete, proprietary and non-transparent ot not suitable for interoperability between heterogeneous systems. 
\par
\bigskip

Another significant issue arises from limitations regarding \textbf{scalability of VANET-based cooperative perception systems}.

One essential challenge is the limited \textbf{throughput}, \textbf{range} and minimum \textbf{latency} of DSRC networks, usually based on IEEE 802.11p technology. \cite{5GAutomotiveAssociation2018} found DSRC to be 90 \% reliable at 675 m distance between sender and receiver in line-of-sight scenarios and 375 m in non-line-of-sight situations. As a comparison, 5G showed 90 \% reliability at 1175 m and 875 m, respectively. Although the absolute results of these tests are debatable, as \cite{Mangel2011} found DSRC NLOS reception to be only 50 \% at 50 m distance, it is clear that 5G technology usually provides better performance. Measurements concerning maximum throughput with IEEE 802.11p range from 2.7 Mbps to 11 Mbps per channel. Average latency in typical CP use cases was measured to range from 3 ms to 22 ms by \cite{Rauch2011}.

Despite limited performance of DSRC, VANET-topology-based CP systems face another problem, which is related to \textbf{network utilization}. Many implementations of such networks implement complete pair-wise P2P connections among participants. Consequently, the number of connections is given as $N = \frac{n(n-1)}{2 }$ according to \textit{Metcalfe's law}. 

Due to these inherent restrictions of DSRC, recent publications tend to favor upcoming 5G technology \cite{Briegleb2019, 5GAutomotiveAssociation2016}, e.g. \cite{Wevers2017} claims that \textit{"'development of V2X services in 5G makes much sense, and holds promises for the future"'}.

A more detailed comparison between DSRC and 5G is presented in \autoref{ch:design_implementaion}.

\section{Traffic Volume Estimation}
\label{sec:problem_analysis:traffic_volume_estimation}

To be able to formulate requirements for a cooperative perception system, a quantification of common traffic volumes is required. More precisely, this section aims to determine an average-case approximation of how many concurrent vehicles will usually be situated within a certain geographical area. Focus is placed on inner-city scenarios. 

\subsection{Methodology}
\label{subsec:problem_analysis:methodology}
An average-case value for the average number of concurrent vehicles within 1 km² of urban area is searched. For its calculation, a heuristic two-step approach is employed. Although based on many assumptions, this rough approximation is sufficient for our purpose as there is no need for a precise quantity. 

The city of Berlin is used as an example, as suitable open data is available for it. OpenStreetMap data files for Berlin \footnote{\url{http://download.geofabrik.de/europe/germany/berlin.html}} served as a basis to compute road network length and average number of street lanes. They were fed into a PostgreSQL \footnote{\url{https://postgresql.org/}} database with the PostGIS \footnote{\url{http://postgis.net/}} extension using \textit{osm2pgsql} \footnote{\url{https://github.com/openstreetmap/osm2pgsql}} and queried using SQL.

\subsubsection{Assumptions}
The following assumptions are made.

\begin{samepage}
\begin{enumerate}
	\item Average inner-city \textbf{driving speed} is 24 km/h \cite{Forbes2008}
	\item Vehicles keep a \textbf{distance} of $s \ [m] = v \ [\frac{m}{sec}] * 1.8 \ [sec] = v \  [\frac{km}{h}] * 0.5 \ [h]$ \cite{wiki:sicherheitsabstand}
	\item Average \textbf{vehicle length} is 4.2 m
	\item \textbf{Area} of Berlin Mitte is 21.25 km² (see appendix \autoref{sec:appendix:source_code:traffic_volume})
	\item Total \textbf{length of road network} in Berlin Mitte is 159.13 km (see appendix \autoref{sec:appendix:source_code:traffic_volume})
	\item Average number of \textbf{driving lanes} in Berlin Mitte is 2.4 (see appendix \autoref{sec:appendix:source_code:traffic_volume})
\end{enumerate}
\end{samepage}

\subsubsection{Step 1: Worst-case density at maximum utilization}
First, we presume a scenario in which the inner-city traffic network is at maximum utilization, that is, all driveable roads are occupied.

Given that every car drives the assumed average speed and keeps its minimum safety distance, the virtual length of every car is: $$l_{virt} = \SI{4.2}{\meter} + (\SI{24}{\meter\per\second} * \frac{1}{3.6} * 1.8) = \SI{4.2}{\meter} + \SI{12}{\meter} = \SI{16.2}{\meter} = \SI{0.0162}{\km}$$.

In addition, the total length of driveable lanes is:
$$l_{road} = \frac{\SI{159.13}{\km}}{\SI{21.25}{\square\km}} * 2.4 = \SI{17.97}{\km\per\square\km}$$

Accordingly, the maximum possible amount of concurrent cars within an area of 1 km² is:
$$N_{max} = \frac{\SI{17.97}{\km}}{\SI{0.0162}{\km}} = \textbf{1109 [\si{cars\per\km}]}$$

\subsubsection{Step 2: Normalization with estimated average load factor}
While the previous value of 1109 concurrent connected cars per km² is a worst-case assumption, Berlin's latest traffic census \cite{VerkehrslenkungBerlinVLB2014} can be used as ground truth for estimating an average \textbf{load factor}. 

\begin{figure}[H]
	\centering
	\includegraphics[width=1.0\linewidth]{98_images/berlin_traffic_map_2}
	\caption{Traffic Census Map for Berlin Mitte \cite{VerkehrslenkungBerlinVLB2014}}
	\label{fig:berlin_traffic_map}
\end{figure}

\autoref{fig:berlin_traffic_map} depicts an excerpt from that traffic census map, where the red rectangle marks the \SI{21.25}{\square\km} area used in this evaluation. Red circle indicate five measuring points that were randomly selected for calculation of the average utilization factor. For each of these points a daily vehicle count is given as $N_{i}$ in $\frac{vehicles}{24 h}$, alongside the number of driving lanes $m_i$ at that location.

\begin{gather*}
N_1 = 32,000; \  m_1 = 6 \\
N_2 = 18,000; \  m_2 = 2 \\
N_3 = 16,000; \  m_3 = 2 \\
N_4 = 8,000; \  m_4 = 2 \\
N_5 = 24,000; \  m_5 = 4
\end{gather*}

For more accurate estimation it would have been desirable to have the amount of vehicles per measuring point be modeled as an in-homogeneous Poisson process. However, data is only given in per-day resolution. 

Given the above assumptions for speed, vehicle length and safety distance, the time that one vehicle takes to pass a measurement point can be calculated: $$\Delta t_{pass} = \frac{\SI{0.0162}{\km}}{\SI{24}{\km\per\hour}} = \SI{0.000675}{\hour} = \SI{2.43}{\second}$$

Accordingly, the maximum number of vehicles that can potentially pass a measuring point in 24 hours depend on the number of lanes at that measuring point and is given as: $$N_{pot\_i} = \frac{\SI{24}{\hour}}{\SI{0.000675}{cars\per\hour}} * m_i$$

For every point, its load- or utilization factor can be calculated as the fraction of actually measured cars and maximum potential cars: $$\lambda_i = \frac{N_i}{N_{pot_i}}$$

For the above points it follows
\begin{gather*}
N_{pot\_1} = 213,333; \  \lambda_1 = 15 \% \\
N_{pot\_2} = 71,111; \  \lambda_2 = 25.3 \% \\
N_{pot\_3} = 71,111; \  \lambda_3 = 22.5 \% \\
N_{pot\_4} = 71,111; \  \lambda_4 = 11.25 \% \\
N_{pot\_5} = 142,220; \  \lambda_5 = 16.9 \% \\
\end{gather*}
and an average load factor of $\lambda_{avg} = 18.19 \%$.

It can be concluded that, given the average load factor, an optimistic estimate for the average number of concurrent cars in Berlin Mitte might be given by $$N_{norm} = N_{max} * \lambda_{avg} = \SI{1109}{cars\per\km} * 0.1819 = \textbf{202 \si{cars\per\km}}$$.

Assuming that all cars were connected and participating in cooperative perception, $N_{norm}$ is the amount of vehicles the system should, at minimum, be able to handle. A more pessimistic value, while neglecting the participation of pedestrians and road infrastructure for simplicity, is given by $N_{max}$.