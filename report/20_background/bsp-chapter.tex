This chapter introduces to essential topics and concepts in the context of this work and provides the reader with background knowledge required to follow in later chapters.

\section{Autonomous Driving}
\label{sec:background:autonomous_driving}

\subsection{Levels of Autonomy}
\label{subsec:background:labels_of_autonomy}
With reference to self-driving cars, a distinction is usually made between five different levels of autonomy \cite{Klein}. These levels are used to commonly describe vehicles' capabilities and their degree of independence from a human driver with regard to the task of driving. 

\begin{itemize}
	\item \textbf{Level 0:} No computer assistance of any kind. A car is completely controlled by its human driver.
	\item \textbf{Level 1:} Basic assistance, e.g. adaptive cruise control. While most functions are controlled by the driver, the car might take responsibility of a single task, e.g. accelerating and decelerating in certain scenarios.
	\item \textbf{Level 2:} Partial automation, e.g. cruise control and lane centering. At this level, a car is able to take over multiple driving tasks in combination. While the driver is still required to monitor the roadway, she is "'disengaged from physically operating the vehicle"' \cite{Klein} and may keep her hands of the steering wheel and feet of the pedals. To ensure that a driver is still pays full attention and is able to intervene in case of system failures or critical situations, various methods of \textit{Driver Monitoring} are employed. Such include to visually observe a driver's face using cameras or to measure the force applied to the steering wheel.
	\item \textbf{Level 3:} High degree of automation. At this level, a driver might fully rely on a car's self-driving under most conditions, delegating all safety-critical function to the ADAS. Usually, it would maintain a comprehensive awareness of its environment and is able to react on it. Although a driver still has to be present and prepared to take occasional control, she is not required to constantly monitor the traffic. 
	\item \textbf{Level 4:} Full automation. This refers to a system that is able to "'perform all safety-critical driving functions and monitor roadway conditions for an entire trip."' \cite{2016transportation} At this level, there is no necessity for a driver to actually occupy the vehicle.Driving performance of Level 4 autonomous cars is at least equal to human level and would generally even surpass it. 
	\item \textbf{Level 5:} Full autonomy. The highest level of automation describes a system that is capable of driving under any conditions, even extreme ones. Its performance is expected to be at least human-like or even surpass human driving capabilities. 
\end{itemize}

\subsection{State of the Art}
\label{subsec:background:state_of_the_art}

\subsection{Sensor Fusion}
\label{subsec:background:sensor_fusion}

\subsection{Autonomous Driving Pipeline}
\label{subsec:background:autonomous_driving_pipeline}
